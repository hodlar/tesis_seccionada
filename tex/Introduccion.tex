\chapter{Introducci\'on}


El estudio del Sol ha resultado indispensable para el desarrollo de la humanidad. En la antig\"uedad, por ejemplo, la observaci\'on del Sol permiti\'o al g\'enero humano estimar la duraci\'on de las estaciones del a\~no; lo cual permiti\'o realizar predicciones, entre muchas otras cosas sobre la agricultura y los mejores tiempos para emprender largos viajes. Hoy en d\'ia, el estudio del Sol est\'a indisolublemente relacionado con el progreso de nuestras tecnolog\'ias. Por ejemplo, se puede decir que la actividad del Sol, particularmente de su atm\'osfera tienen una relaci\'on directa con nuestros avances aeroespaciales y en materia de telecomunicaciones.

El estudio de \'esta resulta importante, pues como ha ocurri\'do con anterioridad, la actividad solar puede incluso llegar a da\~nar infraestructuras en telecomunicaciones, redes el\'ectricas y/o sistemas de geo-posicionamiento global (GPS)\citep{carrington}. Uno de los prominentes rasgos manifestados por la actividad solar son los repentinos y radicales abillantamientos conocidos como r\'fagas solares que son eventos en donde se aceleran cargas el\'ectricas liberando descomunales cantidades de energ\'a de t\'ipicamente d $10^{32}$~ergs en forma radiante, en forma t\'ermica y en forma cin\'etica. A su vez, las r\'afagas solares usualmente anteceden a fen\'omenos de mayor escala conocidos como Eyecciones de Masa Coronal (EMC), que son desprendimientos de vastas porciones de la atm\'osfera externa del Sol y que se proyectan al medio interplanetario, y que eventualmente arriban a la Tierra causando potenciales desastres naturales y tecnol\'ogicos como los descritos en~\citep{carrington}

Es as\'i como el diagn\'ostico del estado f\'isico de la atm\'osfera solar conlleva a importantes cuestionamientos y avances en la descripci\'on del modo en c\'omo se genera y se transporta energ\'ia a trav\'es de \'esta, principalmente en un estado previo a la gestaci\'on de las r\'afagas solares y EMC.

Sin embargo, hasta el momento, a la disciplina de la f\'isica solar le ha resultado imposible desarrollar un modelo preciso que prediga el comportamiento de la atm\'osferma solar. Esto se debe a que a\'un no se posee conocimiento del todo certero de la f\'isica ivolucrada. En principio, existe una multitud de trabajos cuyo argumento para las repentinas y radicales variaciones de la atm\'osfera solar, las cuales son causadas por inestabilidades de los campos magn\'eticos preexistentes \citep{chromotemp}, una suposici\'on basada en fuertes evidencias observacionales. Considerando lo anterior, en esta tesis se basa emp\'iricamente en este aspecto causal a trav\'es de una extensi\'on de un modelo de simulaci\'on computacional.

Asismismo, esta tesis extiende el estado del arte de la f\'isica solar mediante dos principales contribuciones. La primera contribuci\'on reside en proveer evidencia parcial y exploratoria, a trav\'es de una simulaci\'on computacional,  que las variaciones repentinas y radicales de la crom\'osfera se deben al efecto de los campos magn\'eticos de los niveles cromosf\'ericos y coronales. La segunda contribuci\'on consiste en complementar un modelo de simulaci\'on computacional existente para que considere el efecto de los campos magn\'eticos mismos. Con relaci\'on a esta segunda contribuci\'on, se refiere al modelo de simulaci\'on computacional llamado \emph{Pakal}, empleado para diagnosticar la emisi\'on milim\'etrica y submilim\'etricas proveniente de niveles coronales~\citep{2010ApJS..188..437D} con una resoluci\'on espacial comparable con la alcanzada con el instrumental actual.

Actualmente, el c\'odigo de Pakal realiza sus modelos de simulaci\'on sin considerar el efecto de los campos magn\'eticos en el c\'alculo de la densidad del plasma emisor de la crom\'osfera. Esto ocasiona que sus resultados sobre la densidad no sean del todo precisos en comparaci\'on con las observaciones reales. Mediante la extensi\'on del c\'odigo propuesta en esta tesis, se le posibilita a Pakal la capacidad de considerar el efecto de los campos magn\'eticos. Como resultado, sus aproximaciones a la densidad observada son m\'as precisos. Lo anterior permite robustecer el conocimiento de las variaciones repentinas y radicales observables en la temperatura de la crom\'osfera a partir de par\'ametros de la estructura magn\'etica tridimensional.

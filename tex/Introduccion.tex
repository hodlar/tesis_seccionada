\chapter{Introducci\'on}


El estudio del Sol ha resultado indispensable para el desarrollo de la humanidad. En la antig\"uedad, por ejemplo, la observaci\'on del Sol permiti\'o a los humanos generar el modelo de estaciones del a\~no; el cual nos permiti\'o realizar predicciones sobre, entre muchas otras cosas, la agricultura y los mejores tiempos para emprender largos viajes. Hoy en d\'ia, el estudio del Sol est\'a indisolublemente relacionado con el progreso de nuestras tecnolog\'ias. Por ejemplo, se puede decir que la actividad del Sol, particularmente de su atm\'osfera tienen una relaci\'on directa con nuestros avances aeroespaciales y en materia de telecomunicaciones.

El estudio de \'esta resulta importante, pues como ha ocurri\'do con anterioridad, la actividad solar puede incluso llegar a da\~nar nuestro sistemas de telecomunicaciones \citep{carrington}. Principalmente, el mayor desconcierto que existe con el comportamiento de la actividad solar son los repentinos y radicales abillantamientos conocidos como r\'fagas solares que son eventos en donde se aceleran cargas el\'ectricas y se producen descomunales cantidades de energ\'a de t\'ipicamente d $10^{32}$~ergs en forma radiante, en forma t\'ermica y en forma cin\'etica respectivamente. A su vez, las r\'afagas solares usualmente anteceden a fen\'omenos de mayor escala conocidos como Eyecciones de Masa Coronal, que son desprendimientos de grande porciones de la atm\'osfera externa del Sol y que se proyectan al medio interplanetario, donde eventualmente pueden arribar a la Tierra causando potenciales desastres naturales y tecnol\'ogicos como los descritos en~\citep{carrington}

Es as\'i como el estudio del estado f\'isico de la atm\'osfera solar conlleva a importantes cuestionamientos y avances en la descripci\'on del modo en c\'omo se genera y se transporta energ\'ia a trav\'es de \'esta.

Sin embargo, hasta el momento, a la disciplina de la f\'isica solar le ha resultado imposible desarrollar un modelo preciso que prediga el comportamiento de la atm\'osferma solar. Esto se debe a que a\'un no se posee de un conocimiento certero de la f\'isica de la atm\'osfera solar. Algunos te\'oricos han argumentado que las repentinas y radicales variaciones de la atm\'osfera solar son causadas por inestabilidades de los campos magn\'eticos ah\'i presentes \citep{chromotemp}. Considerando lo anterior, en esta tesis se estudia emp\'iricamente dicha teor\'ia a trav\'es de una extensi\'on de un modelo de simulaci\'on computacional.

Precisamente, esta tesis extiende el presente conocimiento en f\'isica solar mediante dos principales contribuciones. La primera contribuci\'on reside en proveer evidencia parcial y exploratoria, a trav\'es de una simulaci\'on computacional,  que las variaciones repentinas y radicales de la crom\'osfera se deben al efecto de los campos magn\'eticos de los niveles cromosf\'ericos y coronales. La segunda contribuci\'on se halla en la extensi\'on de un modelo de simulaci\'on computacional existente para que considere el efecto de los campos magn\'eticos mismos. Con relaci\'on a esta segunda contribuci\'on, se extiende particularmente el modelo de simulaci\'on computacional llamado Pakal empleado para diagnosticar la emisi\'on milim\'etrica y submilim\'etricas proveniente de niveles coronales~\citep{2010ApJS..188..437D}.

Actualmente, el c\'odigo de Pakal realiza sus modelos de simulaci\'on sin considerar el efecto de los campos magn\'eticos en el c\'alculo de la densidad del plasma de la crom\'osfera. Esto ocasiona que sus resultados sobre la densidad no sean del todo precisos en comparaci\'on con las observaciones reales. Mediante la extensi\'on del c\'odigo propuesta en esta tesis, se le posibilita a Pakal la capacidad de considerar el efecto de los campos magn\'eticos. Como resultado, sus aproximaciones a la densidad observada son m\'as precisos. Lo anterior puede ayudar a entender mejor las variaciones repentinas y radicales observables en la temperatura de la crom\'osfera. Y como se mencion\'o anteriormente, sugiere que hay evidencia muy parcial y exploratoria del efecto de los campos magn\'eticos en el comportamiento de la crom\'osfera.

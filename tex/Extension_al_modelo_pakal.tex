\chapter{Extensi\'on al modelo Pakal \-- A\~nadiendo presi\'on magn\'etica}

La extensi\'on propuesta en esta tesis al modelo computacional Pakal le a\~nade a este \'ultimo el efecto de la presi\'on magn\'etica a la ecuaci\'on de estado. Con lo anterior, el modelo de Pakal, que en principio era un modelo computacional que simulaba un modelo hidrost\'atico para el comportamiento del plasma, se robustece al convertirse en un modelo Magnetohidrost\'atico. El modelo original resuelve la ecuaci\'on de estado del gas suponiendo una atm\'osfera est\'atica y estratificada:

\begin{equation} \label{ecuacion_hidrostatica}
\frac{dp}{dz} = -g\rho
\end{equation}

Siendo \textit{p} la presi\'on, \textit{z} la profundidad geom\'etrica, \textit{g} la aceleraci\'on de la gravedad y \textit{$\rho$} la densidad. Para la presi\'on, se toma en cuenta la presi\'on producida por el propio gas en equilibrio hidrost\'atico, m\'as la presi\'on por turbulencia.

\begin{equation} \label{ecuacion_presion}
p = p_g + \frac{1}{2}\rho v_t^2
\end{equation}

En la ecuaci\'on \ref{ecuacion_presion}, el t\'ermino $p_g$ es la presi\'on del gas en equilibrio termodin\'amico ($p_g$ = $n_tk_{\tiny\mbox{B}}T$), mientras que el t\'ermino $v_t$ es definido como la velocidad promedio de turbulencia, la cu\'al caracteriza la presi\'on debida a la turbulencia.

A esta f\'ormula, se le a\~nade el efecto de la presi\'on causada por el campo magn\'etico, conocida como presi\'on magn\'etica ($P_B$), resultando en \citep{priest}: 

\begin{equation} \label{mhs}
p = p_g + \frac{1}{2}\rho v_t^2 + P_B
\end{equation}

El valor utilizado para $P_B$ est\'a dado por $P_B = \frac{B^2}{8\pi}$ donde B es la magnitud del campo magn\'etico en el punto donde se desea realizar el c\'alculo y $\pi\approx 3.1416$ \citep{PBvalue}. 

A continuaci\'on se precisar\'a el c\'odigo de la extensi\'on computacional al programa Pakal realizado en esta tesis. Se contextualizar\'a esta modificaci\'on con secciones del c\'odigo original. Las extensiones al c\'odigo se diferenciar\'an al c\'odigo original al estar con un estilo de letra en \textbf{negrita}, mientras que los nombres de los archivos se encuentran \underline{subrayados}. Adem\'as, se comentar\'a la funci\'on realizada por el c\'odigo agregado. 

Para hacer posible la adici\'on del efecto de la presi\'on magn\'etica sobre la ecuaci\'on de estado dentro del c\'odigo Pakal, se modificaron los siguientes archivos:\newline

\underline{hmodel.h}
En esta librer\'ia se definen la estructura de la atm\'osfera, a la cual se le agregaron las componentes x, y, z del campo magn\'etico, as\'i como la variable $\xi$ para la modulaci\'on del mismo
\begin{lstlisting}[style=CStyle]
typedef struct{
	int id;
	double z;
	double T;
	double P;
	double H;
	double V;
	double vt;
	double ne;
	double ne_lte;
	double bhm;
	double fz;
	/*-------------------------------------------------------------*/
	//Se agregan las componentes del campo y xi
	double Bx;
	double By;
	double Bz;
	double xi;
	/*-------------------------------------------------------------*/	
}Atmosphere;

\end{lstlisting}

\underline{main.c}
\'Esta es la parte donde el m\'odulo Jaguar se encarga de llamar a las respectivas funciones que llevan a cabo los c\'alculos de cada uno de los pasos de integraci\'on. En este c\'odigo se agreg\'o la lectura de los valores de campo.\newline
Los par\'ametros amplitud e intensidad son los respectivos valores del campo magnetico.\newline
El par\'ameto alpha se define en el capitulo 4.\newline
El par\'ametro B\_x0 representa la altura a la que nace el campo magn\'etico, tomando como referencia la base de la fot\'osfera.
\begin{lstlisting}[style=CStyle]

int main(int argc, char **argv){
	int i,j;
	Model model;
	char env[500];
	char comando[500];
	double hydro_step;
	double pz1;
	double Y;
	double fz1;
	double dx;
	int chromospheric_network;
	int cell;
	int hydro;
	/*----------------------------------------------------------*/
	//Los par\'ametros amplitud e intensidad son los respectivos valores del campo magn\'etico
	//El par\'ametro alpha representa el \'angulo entre el nacimiento del campo magn\'etico y el punto que se desea calcular, tomando como referencia de centro el centro solar
	//B_x0 representa la altura de nacimiento del campo, tomando como referencia la base de la fot\'osfera
	//magnetic es la variable que nos dice si se llam\'o la bandera del campo magn\'etico
	double amplitude, intensity, alpha;
	int magnetic = 0;
	int B_x0;
	/*----------------------------------------------------------*/

	printf("Loading Atomic Model:\n");
	model = newModel(argv[1]);
	chromospheric_network=0;
	hydro = 0;

	for (i=1; i<argc; i++) {
		sprintf(comando,"%s",argv[i]);
		if (strcmp(comando,"-hydro") == 0) {
			hydro= 1;
		}

		if (strcmp(comando,"-cn") == 0) {
			sprintf(comando,"%s",argv[++i]);
			if (sscanf(comando,"%i\n",&cell) > 0) {
				chromospheric_network=1;
				printf("Error: Network or cell is required.\n");
				return 0;
			}
		}
		/*-----------------------------------------------------------*/	
		//Aqui se lee la bandera del campo magnetico representada por -B y se transfieren al codigo los parametros de entrada
		if (strcmp(comando, "-B") == 0) {
			sprintf(comando, "%s",argv[++i]);
			if(sscanf(comando,"%lf\n",&amplitude) > 0) {
				sprintf(comando, "%s", argv[++i]);
				if(sscanf(comando,"%lf\n",&intensity) > 0) {
					sprintf(comando, "%s", argv[++i]);
					if(sscanf(comando,"%lf\n",&alpha) > 0) {
						sprintf(comando, "%s", argv[++i]);
						magnetic = 1;
					}
				}
			}
		}
		/*----------------------------------------------------------*/	
	}

	B_x0 = 1;
	loadInitValues(&model, 0);
	if(magnetic == 1) {
		init_B(amplitude, intensity, alpha, &model, B_x0);
	}
	printf("Layer 0\n");
	NLTE(&model,1e-14,0,0.0,0.0,0.0,chromospheric_network,cell,0);
	writeModel(model,"dummy/");

	for (i=1; i<= model.n; i++) {
		pz1 = model.atm.P;
		fz1 = model.atm.fz;
		dx = model.atm.z;
		loadInitValues(&model, i);
		/*----------------------------------------------------------*/	
		//En caso de que la bandera haya sido activada, se calcula el campo magnetico y modidica directamente los valores de las capas de la atmosfera por medio del parametro &model.
		if(magnetic == 1) {
			calculate_B(amplitude, intensity, alpha, &model, B_x0);
		}
		/*----------------------------------------------------------*/	
		if (!(hydro)) {
			printf("Layer %i (ion)\n",i);
			/*---------------------------------------------------------*/	
			//Dado que la opcion hydro no fue activada, se asume que no se busca tampoco el campo magnetico (esto se representa por el 0 del final)
			/*---------------------------------------------------------*/	
			NLTE(&model,1e-14,0,0.0,0.0,0.0,chromospheric_network,cell,0);
		}else{
			printf("Layer %i (hydro)\n",i);
			dx = model.atm.z - dx;
			/*---------------------------------------------------------*/	
			//Si la opcion hydro fue activada, se le pasa al modelo NLTE el parametro "magnetic"
			/*---------------------------------------------------------*/	
			NLTE(&model,1e-14,1,pz1,fz1,dx,chromospheric_network,cell,magnetic);
		}
		writeModel(model,"dummy/");
	}
	return 0;
}
\end{lstlisting}

\underline{nlte.c}
Esta seccion resuelve la ecuaci\'on de estado con un modelo NLTE (non local thermodynamic equilibrium), y es donde se agrega como tal la influencia de la presi\'n magn\'etica
\begin{lstlisting}[style=CStyle]

#include <string.h>

double totalParticles =0.0;
double magnetic_true = 0;

/*---------------------------------------------------------*/
//Es aqui donde se resuelve la funcion de estado, por lo que es donde se realiza la principal contribucion de esta tesis.

double f_hydro(double Y, double R, double Z, double T, double vt, double bx, double by, double bz, double xi){
	//g * m_H 6.674e-8 cm^3g-1s-2 * 1.673534e-24 g
	double R1,KT,R2,R3,R4,R5,R6,B, magnetic_field;


	//Se agrega un archivo que contiene los valores del campo magnetico y de la presion a cada altura, para poder observarlos y graficarlos de ser necesario. Los valores se encuentran contenidos en el archivo magnetic_and_pressure.dat
	char tmp_name[300];
	FILE *tmp_file;
	strcpy(tmp_name,"magnetic_and_pressure.dat");
	//De momento solamente se toma en cuenta la componente z del campo magnetico, por lo que esta se asigna al valor del campo
	magnetic_field = bz;

	if(magnetic_true != magnetic_field)
	{
		tmp_file = fopen(tmp_name,"a");
		magnetic_true = magnetic_field;
		printf("magnetic_field=%le\n",magnetic_field);
		fprintf(tmp_file, "%le   %le\n", 0.0, xi*magnetic_field/(8*M_PI));
		fclose(tmp_file);
	}
	first_time = 1;
	//Esta es la ecuacion resultante con la componente del campo magnetico (xi*magnetic_field/(8*M_PI))
	B = 1.0+R+Y+Z+( (0.5*pow(vt,2.0)*(1.0+4.0*Y)*mH)/(kboltz*T)) + xi*magnetic_field/(8*M_PI);

//B = 1.0+R+Y+Z+( (0.5*pow(vt,2.0)*(1.0+4.0*Y)*mH)/(kboltz*T));
//Esta era la ecuacion antes de ser modificada

	return exp(log(GMH) - log(kboltz*T) + log(1.0+4.0*Y) - log (B));

}
/*---------------------------------------------------------*/

\end{lstlisting}

\underline{nlte.h}
En el encabezado de la librer\'ia de NLTE (non local thermodynamic equilibrium) se modifica para aceptar el campo magnetico
\begin{lstlisting}[style=CStyle]
//Se agregan las componentes x, y, z y xi para ser introducidas en la ecuacion de estado
double f_hydro(double Y, double R, double Z, double T, double vt, double bx, double by, double bz, double xi);
void NLTE(Model *model, double error, int hydro, double pz1, double fz1,double dx, int chromospheric_network,int cell, int magnetic);
\end{lstlisting}

\underline{Pakal.c}
Pakal, al ser la parte del c\'odigo que ``interact\'ua'' con el usuario, es la parte que procesa la informaci\'on de la terminal y la env\'ia al resto del c\'odigo, tomando como entrada de la terminal los valores amplitud, intensidad y alpha, que son respectivamente los valores que se introducir\'an al campo magn\'etico.
Amplitud: La amplitud del campo
Intensidad: La intensidad del campo
Alpha: \'Angulo entre el nacimiento del campo y la l\'inea de visi\'on a la que se desea calcular
\begin{lstlisting}[style=CStyle]
/*---------------------------------------------------------*/
//campo magnetico
double B_amplitude = 0, B_intensity = 0, B_alpha = 0;
int magnetic = 0;
//componente de campo magnetico
if (strcmp(comando,"-B") == 0) {
	sprintf(comando,"%s",argv[++i]);
	if (sscanf(comando,"%lf\n",&B_amplitude) > 0) {
		sprintf(comando,"%s",argv[++i]);
		if (sscanf(comando,"%lf\n",&B_intensity) > 0) {
			sprintf(comando,"%s",argv[++i]);
			if(sscanf(comando,"%lf\n",&B_alpha) > 0) {
				magnetic=1;
			}
			else{
				imprimeInstrucciones();
				return 0;
			}
		}
		else{
			imprimeInstrucciones();
			return 0;
		}
	}
	else{
		imprimeInstrucciones();
		return 0;
	}
}
/*----------------------------------------------------------*/

if (!(compute_ion_profile)) {
	if (hydro) {

		sprintf(comando,"rm data/atmosphere/chromosphere/average/*.dat");
		system(comando);
		printf("Computing hydrostatic atmosphere.\n");

		if(magnetic) {
			sprintf(comando,"jaguar/jaguar %s -hydro -B %le %le %le",atm_model, B_amplitude, B_intensity, B_alpha);
		}
		else{
			sprintf(comando,"jaguar/jaguar %s -hydro",atm_model);
		}

		system(comando);

		printf("Ready\n");
		return 0;
	}else{
		printf("Using previous ion profiles.\n");
	}
}else{
	if (chromosnet) {
		printf("Computing ion profiles CN activated.\n");
		sprintf(comando,"rm data/atmosphere/chromosphere/chromosnet/cell/*.dat");
		system(comando);
		sprintf(modelCell,"%s-CELL",atm_model);
		printf("Computing %s \n", modelCell);
		sprintf(comando,"jaguar/jaguar %s -cn 1",modelCell);   
		printf("%s\n",comando);
		system(comando);
		printf("Ready\n");
		printf("Computing ion profiles.\n");
		sprintf(comando,"rm data/atmosphere/chromosphere/chromosnet/net/*.dat");
		system(comando);
		sprintf(modelNet,"%s-NET",atm_model);
		printf("Computing %s \n", modelNet);
		sprintf(comando,"jaguar/jaguar %s -cn 0",modelNet);
		printf("%s\n",comando);
		system(comando);
		printf("Ready\n");
		return 0;
	}else{
		printf("Computing ion profiles.\n");
		sprintf(comando,"rm data/atmosphere/chromosphere/average/*.dat");
		system(comando);
		sprintf(comando,"jaguar/jaguar %s",atm_model);
		printf("%s\n",comando);
		system(comando);
		printf("Ready\n");
		return 0;
	}
}
\end{lstlisting}
